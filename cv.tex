%------------------------------------
% Dario Taraborelli
% Typesetting your academic CV in LaTeX
%
% URL: http://nitens.org/taraborelli/cvtex
% DISCLAIMER: This template is provided for free and without any guarantee 
% that it will correctly compile on your system if you have a non-standard  
% configuration.
% Some rights reserved: http://creativecommons.org/licenses/by-sa/3.0/
%------------------------------------

%!TEX TS-program = xelatex
%!TEX encoding = UTF-8 Unicode

\documentclass[10pt, a4paper]{article}
\usepackage{fontspec} 

% DOCUMENT LAYOUT
\usepackage{geometry} 
\geometry{a4paper, textwidth=5.5in, textheight=8.5in, marginparsep=7pt, marginparwidth=.6in}
\setlength\parindent{0in}

% FONTS
\usepackage[usenames,dvipsnames]{xcolor}
\usepackage{xunicode}
\usepackage{xltxtra}
\defaultfontfeatures{Mapping=tex-text}
%\setromanfont [Ligatures={Common}, Numbers={OldStyle}, Variant=01]{Linux Libertine O}
%\setmonofont[Scale=0.8]{Monaco}
%%% modified by Karol Kozioł for ShareLaTeX use
\setmainfont[
  Ligatures={Common}, Numbers={OldStyle}, Variant=01,
  BoldFont=LinLibertine_RB.otf,
  ItalicFont=LinLibertine_RI.otf,
  BoldItalicFont=LinLibertine_RBI.otf
]{LinLibertine_R.otf}
\setmonofont[Scale=0.8]{DejaVuSansMono.ttf}

% ---- CUSTOM COMMANDS
\chardef\&="E050
\newcommand{\html}[1]{\href{#1}{\scriptsize\textsc{[html]}}}
\newcommand{\pdf}[1]{\href{#1}{\scriptsize\textsc{[pdf]}}}
\newcommand{\doi}[1]{\href{#1}{\scriptsize\textsc{[doi]}}}
% ---- MARGIN YEARS
\usepackage{marginnote}
\newcommand{\amper{}}{\chardef\amper="E0BD }
\newcommand{\years}[1]{\marginnote{\scriptsize #1}}
\renewcommand*{\raggedleftmarginnote}{}
\setlength{\marginparsep}{7pt}
\reversemarginpar

% HEADINGS
\usepackage{sectsty} 
\usepackage[normalem]{ulem} 
\sectionfont{\mdseries\upshape\Large}
\subsectionfont{\mdseries\scshape\normalsize} 
\subsubsectionfont{\mdseries\upshape\large} 

% PDF SETUP
% ---- FILL IN HERE THE DOC TITLE AND AUTHOR
\usepackage[%dvipdfm, 
bookmarks, colorlinks, breaklinks, 
% ---- FILL IN HERE THE TITLE AND AUTHOR
	pdftitle={Curriculum Vitae},
	pdfauthor={Stefan Pernes},
	pdfproducer={}
]{hyperref}  
\hypersetup{linkcolor=blue,citecolor=blue,filecolor=black,urlcolor=MidnightBlue} 

% DOCUMENT
\begin{document}
{\LARGE Stefan Pernes}\\[0.5cm]
\href{mailto:stefan.pernes@gmail.com}{stefan.pernes@gmail.com}\\
\href{http://stefanpernes.github.io}{http://stefanpernes.github.io}\\[1cm]


\section*{Work Experience}
\noindent
\years{2017-present}Research Engineer, INRIA Paris\\
\years{2016-2017}Research Assistant, IDS Mannheim\\
\years{2014-2016}Research Assistant, University of Würzburg\\
\years{2005-2014}IT Support, Agentur Kunst+ Wien\\
\years{2001-2003}Intern, Siemens AG Wien


\section*{Education}
\noindent
\years{present}\textsc{PhD} in Digital Philology, University of Würzburg\\
\years{2013}\textsc{MA} in Applied Linguistics, University of Vienna\\
\years{2010}\textsc{BA} in Journalism and Communication Studies, University of Vienna


\section*{Publications}

\subsection*{Journal articles}
\noindent
\years{2016}Jannidis, F., Reimers, N., Pernes, S., Pielström, S., Vitt, T. (2016). DARIAH-DKPro-Wrapper Output Format (DOF) Specification. DARIAH-DE Working Paper. urn:nbn:de:gbv:7-dariah-2016-6-2\\
%\vskip 0.0005cm
\years{2016}Pernes, S., Pielström, S., Bock, S., Du K., Huber, M. (2016). Der Einsatz quantitativer Textanalyse in den Geisteswissenschaften: Bericht über den Stand der Forschung. DARIAH-DE Working Paper. urn:nbn:de:gbv:7-dariah-2016-4-0\\
\years{2016}Aurast, A., Gradl, T., Pernes, S., and Pielström, S. (2016). Big Data und Smart Data in den Geisteswissenschaften. In: Bibliothek - Forschung und Praxis, Band 40, Heft 2: pp. 200-206, doi:10.1515/ bfp-2016-0033\\
\years{2016}Schumacher, M., Held, M., Falk, C., and Pernes, S. (2016). Big Data in den Geisteswissenschaften: Konzept für eine Lehr- und Lernmittelsammlung. DARIAH-DE Working Papers Nr. 15. Göttingen: DARIAH-DE. urn:nbn:de:gbv:7-dariah-2016-1-2\\
\years{2013}Pernes, S. (2013). Die große Freiheit der kleinen Bücher. In: Schulheft, No.151. pp. 87-92. Innsbruck: StudienVerlag.

\subsection*{Conference Abstracts}
\noindent
\years{2017}Bowers, J., Pernes, S., Romary, L. (2017). conceptEntry: A TBX-based expansion of the TEI for the encoding of onomasiological and comparative lexical data. In TEI Members Meeting 2017: Conference Abstracts. University of Victoria, Victoria BC.\\
\years{2017}Pernes, S., Romary, L., Warburton, K. (2017). TBX in ODD: Schema-agnostic specification and documentation for TermBase eXchange. In Proceedings of Language, Ontology, Terminology and Knowledge Structures Workshop (LOTKS 2017). ACL Anthology W17-7000\\
\years{2017}Pernes, S., Keller, L., Peterek, C. (2017). Aufbau eines historisch-literarischen Metaphernkorpus für das Deutsche. In DHd 2017, Digital Humanities im deutschsprachigen Raum: Konferenzabstracts. Universität Bern, Bern, pp. 91-94.\\
\years{2016}Pernes, S. (2016). Metaphor Mining in Historical German Novels: Using Unsupervised Learning to Uncover Conceptual Systems in Literature. In Digital Humanities 2016: Conference Abstracts. Jagiellonian University and Pedagogical University, Kraków, pp. 651-653.\\
\years{2016}Reimers, N., Jannidis, F., Pernes, S., Pielström, S., Reger, I., Vitt, T. (2016). A Tool for NLP-Preprocessing in Literary Text Analysis. In Digital Humanities 2016: Conference Abstracts. Jagiellonian University and Pedagogical University, Kraków, pp. 871-872.\\
\years{2015}Pernes, S. (2015). Metaphor Mining in Historical German Novels: An Unsupervised Learning Approach. In: Proceedings of the IEEE International Conference on Big Data 2015, Santa Clara, pp. 1650-1652. doi:10.1109/BigData.2015.7363934

\subsection*{Book Chapters}
\noindent
\years{2014}Schreger, C. and Pernes, S. (2014). The Big World of ‚Little Books’. In Hélot, C., Sneddon, R. and Daly, N. (eds). Children's Literature in the Multilingual Classroom. London: IOE Press, pp. 154-171

\subsection*{Teaching Materials}
\noindent
\years{2015}Pielström, S., Pernes, S., Reimers, N., Bock, S., Dürholt, P., Du, K. (2015): NLP Based Analysis of Literary Texts. \href{http://dariah-de.github.io/DARIAH-DKPro-Wrapper/tutorial.html}{http://dariah-de.github.io/DARIAH-DKPro-Wrapper/tutorial.html}


\section*{Talks}
\years{2017}conceptEntry: A TBX-based expansion of the TEI for the encoding of onomasiological and comparative lexical data. TEI Members Meeting and Linguistics SIG Meeting, 13./14.11.2017, Victoria BC\\
\years{2017}TBX in ODD: Schema-agnostic specification and documentation for TermBase eXchange. Language, Ontology, Terminology and Knowledge Structures Workshop (LOTKS 2017), 19.09.2017, Montpellier\\
\years{2017}conceptEntry. A TBX-based expansion of the TEI for the encoding of onomasiological and comparative lexical data. ISO/TC 37 Annual Meeting, 29.06.2017, Vienna\\
\years{2017}TBX in TEI. A TBX-based expansion of the TEI for the encoding of onomasiological and comparative lexical data. ALMAnaCH Kick-off Meeting, 18.05.2017, Berlin\\
\years{2017}Aufbau eines historisch-literarischen Metaphernkorpus für das Deutsche. DHd 2017, Digital Humanities im deutschsprachigen Raum, 15.02.2017, Bern\\
\years{2016}Aufbau eines historisch-literarischen Metaphernkorpus für das Deutsche. Stuttgart Research Center for Text Studies, 16.11.2016, Stuttgart (Invited Talk)\\
\years{2016}Metaphor Mining in Historical German Novels: Using Unsupervised Learning to Uncover Conceptual Systems in Literature. Digital Humanities 2016, 14.07.2016, Kraków\\
\years{2016}A Tool for NLP-Preprocessing in Literary Text Analysis. Digital Humanities 2016, 11.-16.07.2016, Kraków (Poster)\\
\years{2016}A Tool for NLP-Preprocessing in Literary Text Analysis. DHd 2016, Digital Humanities im deutschsprachigen Raum, 07.-12.03.2016, Leipzig (Poster)\\
\years{2016}A Tool for NLP-Preprocessing in Literary Text Analysis. DARIAH-DE Grand Tour, 18.-19.02.2016, Göttingen (Poster)\\
\years{2015}Readability Measures. Workshop 'Complexity Measures in Stylometry'. DARIAH-DE Expertenworkshop, 07.12.2015, Würzburg\\
\years{2015}Metaphor Mining in Historical German Novels: An Unsupervised Learning Approach. 3rd Workshop on Big Humanities Data, IEEE International Conference on Big Data 2015, 29.10.2015, Santa Clara\\
\years{2015}Natural Language Processing zur Analyse literarischer Texte. Workshop Natural Language Processing für Literaturwissenschafter. DARIAH-DE Methodenworkshop, 16.09.2015, Würzburg\\
\years{2015}Introduction of DARIAH-EU Working Group 'Text and Data Analytics'. DARIAH-EU 5th General VCC meeting, 22.04.2015, Ljubljana\\
\years{2015}Big, complex, heterogeneous.. Laufende Projekte aus dem Arbeitsbereich 'Big Data in den Geisteswissenschaften' in DARIAH-DE. Digital Humanities Summit 2015, 03.-04.03.2015, Berlin (Poster)\\
\years{2015}Big, complex, heterogeneous.. Laufende Projekte aus dem Arbeitsbereich 'Big Data in den Geisteswissenschaften' in DARIAH-DE. DHd 2015, Digital Humanities im deutschsprachigen Raum, 23.-27.02.2015, Graz (Poster)


\section*{Teaching}
\years{2015}Webdesign and Typesetting, University of Würzburg

\section*{Software and Corpora}
\years{}German Metaphor Corpus (in preparation)\\
\years{2017}TextGrid Digitale Bibliothek TEI-I5 Version:\\
\href{http://www1.ids-mannheim.de/direktion/kl/projekte/korpora.html}{http://www1.ids-mannheim.de/direktion/kl/projekte/korpora.html}\\
\years{2016}DARIAH-DKPro Wrapper: \href{http://dariah-de.github.io/DARIAH-DKPro-Wrapper}{http://dariah-de.github.io/DARIAH-DKPro-Wrapper}

\section*{Service}
\years{}Peer Review: DH2017, ACL2017, CoNLL2017, DH2018\\
\years{}Committee Work: ISO TC37/SC3, DIN-Normenausschuss Terminologie (NAT)


\vfill{}

\begin{center}
{\scriptsize  Last updated: \today\-\- }
\end{center}

\end{document}